\documentclass[12pt]{article}
\usepackage{hyperref}
\title{47 DAYS RF WIZARD}
\author{Shoaib Mustafa}
\begin{document}
\maketitle

\begin{center}
    \section*{DAY 2}
\end{center}
\subsection*{Summary}
Seems like my lack of knowledge of differential equations and classical mechanics is limiting my understanding.
I shall also start to cover a book on differential equations that will teach me how to identify
, characterize and solve and what their effects in the real world are. I've already gone through
the texts of differential equations. What i need is something that will relate it to the real life.

I'm also discarding the book i started with by George Hefkins, instead will be switching to Georgi, Howard.
As suggested in the course by mit. The Hefkins one is nice and really resonates with me but since the i'll be following the course i can't make this my main.
I realize that the course work by mit requires alot of reading and with my final year project and life it's a bit hard to read up on course daily. so as mentioned in the document i'll be meeting with professor 
2 times a week only. the rest of the time i'll be busy with reading. 
\subsection*{Todays Agenda}
\begin{itemize}
    \item watch first video of mit 8.03 and read the relevent texts while also taking notes.
    \item read up on the books of arrl, waves, emt and photonics. to get basic understanding of what a SHM is.. review DE if needed.
    \item create a post with complete summary and update github.
    \item read up on rtl sdr some more to get it working for HF.
\end{itemize}

\subsection*{8.03 Lecture 1}
\subsubsection*{How to take the course as mentioned by professor.}
How Student Time Was Spent\\
During an average week, students were expected to spend 12 hours on the course, roughly divided as follows:
\subsubsection*{In Class}
Met 2 times per week for 1.5 hours per session; 27 sessions total.\\
Class sessions included lectures, demonstrations, and time to ask the professor questions.\\
\subsubsection*{Recitation}
Met 2 times per week for 1 hour per session; 28 sessions total.\\
Smaller groups of students met with class staff to work through problem sets and to ask questions about the course material.\\
\subsubsection*{Out of Class}
Outside of class, students worked on problem sets and prepared for exams.\\

\subsubsection*{George Bekefi text}
Correction: The site 8.03 seems to be following georgi, howard book. i'll follow this new book from tomorrow. \\

This book starts with simple harmonic motion and the idea that such a motion is pretty generlised
and is found in many different fields of physics, such as mechanical, optics, electromagentism etc.
The recipe for the SHM is defined to be following
\begin{itemize}
    \item It must be oscillating (repeating motion) about it's equilibrium position (at which point no net force acts on the system).
    \item It must have a restoring force that is acting opposite to the direction of motion and should be propotional to the force being applied.
    \item It must have maintain periodic motion.
\end{itemize}

It is highlighted that if something shows SHM then it's properties that are a function of time
can be predicted with it's past, present and future known. contrary to my previous belief the property
could be anything, current in an electric oscillator, position. This is why that in classical mechanics
the property under study in present as a generalised 'coordinate' instead of assigning it a name.

To prove this point it starts with a simple scenario of a spring mass on a frictionless surface.
Imagining that force is applid to displace it from the equilibirium position $ X_o $
Now let's keep in mind the general concept, this displacement could be a disturbance in the atom's position,
it's electric field, or magnetic field probably? i'm thinking that this should apply generally.

\begin{quote}
    "If some parameter of the system designating its displacement from the equilibrium position is characterized b ya time-dependent
    coordinate $\psi(t)$" -George Bekefi page-2
\end{quote}
I'm getting that this means that any property that we are measuring is dependent on time $t$ $\psi(t)$ then for that case Newton's law tells us that
\begin{center}
    \begin{equation}
        F = ma = m \frac{d^2 \psi(t)}{dt^2}
    \end{equation}
\end{center}
In this case the time dependent coordiante a is time dependent as it is acceleration which is a derivative of velocity w.r.t time. so what does that tell us anyway?
not sure really. I think will need to refer to classical mechanics for this.

continuing with the scenario it proceeds to say that the net force is the restoring force and so must be given by.
\begin{equation}
    F = -\beta \psi(t)
\end{equation}
where beta is the proportionality constant and it is with experiments we can determine it's value to be specific to the application we are studying,
in the case of mechanical springs it's the springs restoring force, in the case of atoms it's the recoil energy.

The restoring force now contains all the points we talked about. it's proportional and opposite to the direction of motion. in this case the property is position, changing w.r.t time.

combining both equations we get

\begin{equation}
    -\beta \psi(t)  = m \frac{d^2 \psi(t)}{dt^2}
\end{equation}
\begin{equation}
    \beta \psi(t)  + m \frac{d^2 \psi(t)}{dt^2} = 0
\end{equation}
\begin{equation}
    \frac{d^2 \psi(t)}{dt^2} + \omega^2_o \psi(t) = 0
\end{equation}
\begin{center}
    where $ \omega^2_o = \frac{\beta}{m}$
\end{center}

We are introducing $ \omega^2_o$ as in all our applications there wil be some some restoring force, and intertial term such as inductance, mass etc.
In fact that each application distinguishes from others is by this ratio and terms. this is why we have a generlised SHM equation. and whenever we are asked to solve a problem our main aim is to solve this ration, rest is the same. \\

And it is proven to us that the solution to this equation is also generlised (the only difference being the frequency term.)

if we solve this like any other general 2nd order differential equation.
it's linear, homogenous, constant coefficients.
by t
\begin{center}
    $m^2 + \omega^2_o = 0$ ---$>$
    $m^2 = - \omega^2_o$ \\
    $m = \iota {\omega_o}$ \\
    $\psi(t) = \exp(\iota \omega_o t)$

\end{center}

where the solution is always $\psi(t) = \cos(\omega_o t)$
% TODO: Im not too confident about my DE knowledge. forgotten. need to revise. 

The solution is verifiable by substituting into eq:5\\

In the case of this mechanical spring the $\psi(t)$ represents the displacement $x(t)$
of the mass $m$ with the final equation becomnig;
\begin{center}

    $\frac{d^2x}{dt^2} + \omega^2_o x = 0$

\end{center}
the only difference in this application would be $\omega^2_o = \frac{K}{M}$
where K is the restoring force defined by the spring used.


\subsection*{ARRL}
I learned that in real life though waves are radially propagating at any instant over some distance after the wave is generated from an antenna looks like a flat surface to us, which is why call that plane wave. 
the current oscillation within the antenna essentially creates electormagnetic field which is the thing that propagates. 
The energy is confined to the couple field, and all three namely, propagation direction, Electric field and magnetic field are all orthogonal to each other. why do we talk about the direction that they are facing? 
It helps with understanding the polarization of waves, if they are linear, circular, circular left or circular right. This is a topic i haven't read about much. It also helps with finding out the direction of propagation. 
But i thought i should mention it. 

Though i'm just throwing this out there but i've read that antenna's received and the transmitted waves must both be polarized similarly. i like to think of it this way. 
Imagine a grater with just one slit. if you put a carrot the size of the slit and in the orientation of the slit the carrot will pass with complete mass intact,
higher energy transfer. but if the orientiation of the carrot is different or if the size (frequency) is different it will cut back some of the mass hence the energy. 
so in order for the antenna to receive or transmit properly these properties must be taken into account.

Polarization depends on the plane of propagation of the electric field. and it is this field that we are manipulating. 

Phase wasn't so nicely explained. the book says if i correctly understand that when the wave is similar in all the area of the wave front then it is in phase.. however i find it hard to believe that a certain wavefront could be manipulated such that one part is different while other is not.
i find that phase to be the difference between each successive wavefront being generated. maybe i just misunderstood the book. 

\begin{quote}
    "Phase and Wavelength

Because the speed at which radio waves travel is so great, we are likely to fall into the habit of ignoring the time that elapses between the instant at which a wave leaves the transmitting antenna and the instant at which it arrives at the receiving antenna. It is true that it takes only oneseventh of a second for a wave to travel around the world, and from a communication standpoint that is hardly worth worrying about. But there is another consideration that makes this factor of time extremely important.

The wave is brought into existence because an alternating current flowing in a conductor (which is usually an antenna) sets up the necessary electric and magnetic fields. The alternating currents used in radio work may have frequencies anywhere from a few tens of thousands to several billion cycles per second. Suppose that the frequency is 30 megacycles per second that is, 30,000,000 cycles per second. One of those cycles will be completed in 1/30,000,000 second, and since the wave is traveling at a speed of 300,000,000 meters per second it will have moved only 10 meters during the time that the current is going through one complete cycle. To put it another way, the electromagnetic field 10 meters away from the antenna is caused by the current that was flowing in the antenna one cycle earlier in time; the field 20 meters away is caused by the current that was flowing two cycles earlier, and
so on.

Now if each cycle of current is simply a repetition of the one that preceded it, the currents at corresponding instants in each cycle will be identical, and the fields caused by those identical currents also will be identical. As the fields move outward they become more thinly spread over larger and larger surfaces, so their amplitudes decrease with distance from the antenna. But they do not lose their identity with respect to the instant of the cycle at which they were generated. That is, the phase of the outwardly-moving surface remains constant. It follows, then, that at intervals of 10 meters (in the example above) measured outward from the antenna the phase

of the waves at any given instant is identical."
\end{quote}
it reaches the conclusion that "if we measure the outward from the antenna the phase of the waves at any given instant is identical" 
I don't quite get this. 
I feel like this concept can be better explained from the physics book than in ARRL.
It also gives a novel definition of the wave front: "The wave front is simply a surface in every part of which the wave is in the same phase."\\
however it gives a much better definition of wavelength: "the wavelength is simply the distance between two wave fronts having identical phase at any given instant." 
\\Each wavefront represent a full cycle of current oscillation in the circuit.. producing EM fields. each wavefront will have to be formed from one complete oscillating motion of the current. and hence the wavelenght is the distance of one wave cycle. or from wave front to wave front. i like this definition. 
\\It further explains the concept of phase by saying "phase means time, but when something goes through periodic variations with time in the way that an alternating current does, corresponding instants in succeeding cycles are said to have the same phase even though the actual time difference is one cycle."
\\So what i understand from this is that the phase though talking about different wavefronts or oscillation cycle, it's the quantitative measure of how much it differes from the previous wave. 
\\Further the book compares it with using 24 hour time. if we say 4 O clock, it means 4 O clock today or tomorrow. but the time doesn't change. only the instance is. 


\end{document}