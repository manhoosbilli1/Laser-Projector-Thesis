\documentclass[12pt]{article}
\usepackage{hyperref}
\title{47 DAYS RF WIZARD}
\author{Shoaib Mustafa}
\begin{document}
\maketitle

\begin{center}
    \section*{DAY 2}
\end{center}
\subsection*{Summary}
Seems like my lack of knowledge of differential equations and classical mechanics is limiting my understanding.
I shall also start to cover a book on differential equations that will teach me how to identify
, characterize and solve and what their effects in the real world are. I've already gone through
the texts of differential equations. What i need is something that will relate it to the real life.

I'm also discarding the book i started with by George Hefkins, instead will be switching to Georgi, Howard.
As suggested in the course by mit. The Hefkins one is nice and really resonates with me but since the i'll be following the course i can't make this my main.
I realize that the course work by mit requires alot of reading and with my final year project and life it's a bit hard to read up on course daily. so as mentioned in the document i'll be meeting with professor 
2 times a week only. the rest of the time i'll be busy with reading. 
\subsection*{Todays Agenda}
\begin{itemize}
    \item watch first video of mit 8.03 and read the relevent texts while also taking notes.
    \item read up on the books of arrl, waves, emt and photonics. to get basic understanding of what a SHM is.. review DE if needed.
    \item create a post with complete summary and update github.
    \item read up on rtl sdr some more to get it working for HF.
\end{itemize}

\subsection*{8.03 Lecture 1}
\subsubsection*{How to take the course as mentioned by professor.}
How Student Time Was Spent\\
During an average week, students were expected to spend 12 hours on the course, roughly divided as follows:
\subsubsection*{In Class}
Met 2 times per week for 1.5 hours per session; 27 sessions total.\\
Class sessions included lectures, demonstrations, and time to ask the professor questions.\\
\subsubsection*{Recitation}
Met 2 times per week for 1 hour per session; 28 sessions total.\\
Smaller groups of students met with class staff to work through problem sets and to ask questions about the course material.\\
\subsubsection*{Out of Class}
Outside of class, students worked on problem sets and prepared for exams.\\

\subsubsection*{George Bekefi text}
Correction: The site 8.03 seems to be following georgi, howard book. i'll follow this new book from tomorrow. \\

This book starts with simple harmonic motion and the idea that such a motion is pretty generlised
and is found in many different fields of physics, such as mechanical, optics, electromagentism etc.
The recipe for the SHM is defined to be following
\begin{itemize}
    \item It must be oscillating (repeating motion) about it's equilibrium position (at which point no net force acts on the system).
    \item It must have a restoring force that is acting opposite to the direction of motion and should be propotional to the force being applied.
    \item It must have maintain periodic motion.
\end{itemize}

It is highlighted that if something shows SHM then it's properties that are a function of time
can be predicted with it's past, present and future known. contrary to my previous belief the property
could be anything, current in an electric oscillator, position. This is why that in classical mechanics
the property under study in present as a generalised 'coordinate' instead of assigning it a name.

To prove this point it starts with a simple scenario of a spring mass on a frictionless surface.
Imagining that force is applid to displace it from the equilibirium position $ X_o $
Now let's keep in mind the general concept, this displacement could be a disturbance in the atom's position,
it's electric field, or magnetic field probably? i'm thinking that this should apply generally.

\begin{quote}
    "If some parameter of the system designating its displacement from the equilibrium position is characterized b ya time-dependent
    coordinate $\psi(t)$" -George Bekefi page-2
\end{quote}
I'm getting that this means that any property that we are measuring is dependent on time $t$ $\psi(t)$ then for that case Newton's law tells us that
\begin{center}
    \begin{equation}
        F = ma = m \frac{d^2 \psi(t)}{dt^2}
    \end{equation}
\end{center}
In this case the time dependent coordiante a is time dependent as it is acceleration which is a derivative of velocity w.r.t time. so what does that tell us anyway?
not sure really. I think will need to refer to classical mechanics for this.

continuing with the scenario it proceeds to say that the net force is the restoring force and so must be given by.
\begin{equation}
    F = -\beta \psi(t)
\end{equation}
where beta is the proportionality constant and it is with experiments we can determine it's value to be specific to the application we are studying,
in the case of mechanical springs it's the springs restoring force, in the case of atoms it's the recoil energy.

The restoring force now contains all the points we talked about. it's proportional and opposite to the direction of motion. in this case the property is position, changing w.r.t time.

combining both equations we get

\begin{equation}
    -\beta \psi(t)  = m \frac{d^2 \psi(t)}{dt^2}
\end{equation}
\begin{equation}
    \beta \psi(t)  + m \frac{d^2 \psi(t)}{dt^2} = 0
\end{equation}
\begin{equation}
    \frac{d^2 \psi(t)}{dt^2} + \omega^2_o \psi(t) = 0
\end{equation}
\begin{center}
    where $ \omega^2_o = \frac{\beta}{m}$
\end{center}

We are introducing $ \omega^2_o$ as in all our applications there wil be some some restoring force, and intertial term such as inductance, mass etc.
In fact that each application distinguishes from others is by this ratio and terms. this is why we have a generlised SHM equation. and whenever we are asked to solve a problem our main aim is to solve this ration, rest is the same. \\

And it is proven to us that the solution to this equation is also generlised (the only difference being the frequency term.)

if we solve this like any other general 2nd order differential equation.
it's linear, homogenous, constant coefficients.
by t
\begin{center}
    $m^2 + \omega^2_o = 0$ ---$>$
    $m^2 = - \omega^2_o$ \\
    $m = \iota {\omega_o}$ \\
    $\psi(t) = \exp(\iota \omega_o t)$

\end{center}

where the solution is always $\psi(t) = \cos(\omega_o t)$
% TODO: Im not too confident about my DE knowledge. forgotten. need to revise. 

The solution is verifiable by substituting into eq:5\\

In the case of this mechanical spring the $\psi(t)$ represents the displacement $x(t)$
of the mass $m$ with the final equation becomnig;
\begin{center}

    $\frac{d^2x}{dt^2} + \omega^2_o x = 0$

\end{center}
the only difference in this application would be $\omega^2_o = \frac{K}{M}$
where K is the restoring force defined by the spring used.


\end{document}